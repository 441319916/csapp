%%COMMENT:2:8:Floating point classification:
\begin{problem}{8}

The following procedure takes a single-precision floating point number
in IEEE format and prints out information about what category of
number it is.  Fill in the missing code so that it performs this
classification correctly.

{\large
\begin{alltt}
void classify_float(float f)
\{
  {\it /* Unsigned value {\tt u} has same bit pattern as {\tt f} */}
  unsigned u = *(unsigned *) &f;
  {\it /* Split {\tt u} into the different parts */}
  int sign = (u >> 31) & 0x1;    {\it // The sign bit}

  int exp  = _______________;    {\it // The exponent field}

  int frac = _______________;    {\it // The fraction field}

  {\it /* The remaining expressions can be written in terms of the
     values of {\tt sign}, {\tt exp}, and {\tt frac} */}

  if (______________________)
    printf("Plus or minus zero\verb@\@");

  else if (______________________)
    printf("Nonzero, denormalized\verb@\@");

  else if (______________________)
    printf("Plus or minus infinity\verb@\@");

  else if (______________________)
    printf("NaN\verb@\@");

  else if (______________________)
    printf("Greater than -1.0 and less than 1.0\verb@\@");

  else if (______________________)
    printf("Less than or equal to -1.0\verb@\@");
  else
    printf("Greater than or equal to 1.0\verb@\@");
\}
\end{alltt}
}

% Answers
% (u >> 23) & 0xFF
% u & 0x7FFFFF
% (exp == 0 && frac == 0)
% (exp == 0)
% (exp == 0xFF && frac == 0)
% (exp == 0xFF)
% (exp <= 126)
% (sign == 1)
\end{problem}