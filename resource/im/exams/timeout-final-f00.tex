%%COMMENT:8:8:Process control and signals (timeout version of fgets):
\subsection*{Process control} 

The next problem concerns the following four versions of the {\tt
tfgets} routine, a timeout version of the Unix {\tt fgets} routine.

The {\tt tfgets} routine waits for the user to type in a string and
hit the return key. If the user enters the string within 5 seconds,
the {\tt tfgets} returns normally with a pointer to the
string. Otherwise, the routine ``times out'' and returns a NULL
string.


\subsection*{{\tt tfgets}: Version A}


{\small
\begin{verbatim}
void handler(int sig) {
  siglongjmp(env, 1);
}

char *tfgets(char *s, int size, FILE *stream) {
  pid_t pid;
  signal(SIGCHLD, handler);

  if (!sigsetjmp(env, 1)) {
    pid = fork();
    if (pid == 0) {
      return fgets(s, size, stream);
    }
    else {
      sleep(5);
      kill(pid, SIGKILL);
      wait(NULL);
      return NULL;
    }
  }
  else {
    wait(NULL);
    exit(0); 
  }
}
\end{verbatim}}
% incorrect: gets the parent and the child mixed up


\newpage
{\small
\subsection*{{\tt tfgets}: Version B}
\begin{verbatim}
void handler(int sig) {
  wait(NULL);
  siglongjmp(env,1);
}

char *tfgets(char *s, int size, FILE *stream) {
  pid_t pid;
    
  signal(SIGUSR2, handler);
  if (sigsetjmp(env, 1) != 0)
    return NULL;
  if ((pid = fork()) == 0) {
    sleep(5);
    kill(getppid(), SIGUSR2);
    exit(0);
  }
  fgets(s, size, stream);
  kill(pid, SIGKILL);
  wait(NULL);
  return s;
}
\end{verbatim}}
% OK

\subsection*{{\tt tfgets}: Version C}
{\small
\begin{verbatim}
void handler(int sig) {
  wait(NULL);
  siglongjmp(env, 1);
}

char *
tfgets(char *s, int size, FILE *stream) {
  pid_t pid;
  str = NULL;
  signal(SIGCHLD, handler);

  if ((pid = fork()) ==  0) { 
    sleep(5);
    exit(0);
  }
  else { 
    if (sigsetjmp(env, 1) == 0) {
      str = fgets(s, size, stream);
      kill(pid, SIGKILL);
      pause();
    }
    return str;
  }
}
\end{verbatim}}
% OK

\newpage
\subsection*{{\tt tfgets}: Version D}

{\small
\begin{verbatim}
void handler(int sig) {
  wait(NULL);
  siglongjmp(env, 1);
}

char *
tfgets(char *s, int size, FILE *stream) {
  pid_t pid;
  str = NULL;
  signal(SIGCHLD, handler);

  if ((pid = fork()) ==  0) {
    sleep(5);
    return NULL;
  }
  else {
    if (sigsetjmp(env, 1) == 0) {
      str = fgets(s, size, stream);
      kill(pid, SIGKILL);
      pause();
    }
    return str;
  }
}
\end{verbatim}}
% incorrect: returns from child instead of exiting

\newpage
\begin{problem}{8}
This problem concerns the four versions of {\tt tfgets} from the
previous pages. Some of them are correct,  and others are 
flawed because the author didn't understand basic concepts of
concurrency and signaling. 

Circle the versions that are correct, in the sense that they return
the input string if typed within 5 seconds, timeout after 5
seconds by returning NULL, and correctly reap their terminated
children.
\end{problem}

\begin{verbatim}
Version A         Version B         Version C         Version D         
\end{verbatim}

Note: The {\tt pause} function sleeps until a signal is
received and then returns.

%Answer: B and C are correct
% A gets the parent and child confused.
% D returns from the child instead of exiting.
