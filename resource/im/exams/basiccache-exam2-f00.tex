%%COMMENT:6:5:Low-level cache access:
\begin{problem}{5}
The following problem concerns basic cache lookups.
\end{problem}

\begin{itemize}
\item The memory is byte addressable.
\item Memory accesses are to {\bf 1-byte words} (not 4-byte words).
\item Physical addresses are 13 bits wide.
\item The cache is 2-way set associative, with a 4 byte line size and 16 total lines.
\end{itemize}

In the following tables, {\bf all numbers are given in hexadecimal}.
The contents of the cache are as follows:

\begin{center}
\begin{tabular}{|c||c c|c c c c||c c|c c c c|}
\hline
\multicolumn{13}{|c|}{2-way Set Associative Cache}\\
\hline
\makebox[.2in]{Index} & \makebox[.25in]{Tag} & \makebox[.2in]{Valid} &
\makebox[.3in]{Byte 0} & \makebox[.3in]{Byte 1} &
\makebox[.3in]{Byte 2} & \makebox[.3in]{Byte 3} &
\makebox[.25in]{Tag} & \makebox[.2in]{Valid} &
\makebox[.3in]{Byte 0} & \makebox[.3in]{Byte 1} &
\makebox[.3in]{Byte 2} & \makebox[.3in]{Byte 3} \\ 
\hline
\hline
0  &  09 & 1 & 86 & 30 & 3F & 10  &  00 & 0 & 99 & 04 & 03 & 48 \\
1  &  45 & 1 & 60 & 4F & E0 & 23  &  38 & 1 & 00 & BC & 0B & 37 \\
2  &  EB & 0 & 2F & 81 & FD & 09  &  0B & 0 & 8F & E2 & 05 & BD \\
3  &  06 & 0 & 3D & 94 & 9B & F7  &  32 & 1 & 12 & 08 & 7B & AD \\
4  &  C7 & 1 & 06 & 78 & 07 & C5  &  05 & 1 & 40 & 67 & C2 & 3B \\
5  &  71 & 1 & 0B & DE & 18 & 4B  &  6E & 0 & B0 & 39 & D3 & F7 \\
6  &  91 & 1 & A0 & B7 & 26 & 2D  &  F0 & 0 & 0C & 71 & 40 & 10 \\
7  &  46 & 0 & B1 & 0A & 32 & 0F  &  DE & 1 & 12 & C0 & 88 & 37 \\
\hline
\end{tabular}
%\end{minipage}
\end{center}

\section*{Part 1}

The box below shows the format of a physical address.  Indicate
(by labeling the diagram) the fields that would be used to determine
the following:\\
\begin{tabular}{cl}
{\em CO} & The block offset within the cache line\\
{\em CI} & The cache index\\
{\em CT} & The cache tag\\
\end{tabular}

\vspace{0.2in}
{\small
\begin{tabular} {ccccccccccccc}
\makebox[.15in]{12} & 
\makebox[.15in]{11} & \makebox[.15in]{10} &
\makebox[.15in]{9} & \makebox[.15in]{8} & 
\makebox[.15in]{7} & \makebox[.15in]{6} & 
\makebox[.15in]{5} & \makebox[.15in]{4} & 
\makebox[.15in]{3} & \makebox[.15in]{2} & 
\makebox[.15in]{1} & \makebox[.15in]{0} \\ 
\end{tabular} 
}

\begin{tabular} {|c|c|c|c|c|c|c|c|c|c|c|c|c|}
\hline
\makebox[.15in]{} & \makebox[.15in]{} & \makebox[.15in]{} & 
\makebox[.15in]{} & \makebox[.15in]{} & \makebox[.15in]{} & \makebox[.15in]{} & 
\makebox[.15in]{} & \makebox[.15in]{} & \makebox[.15in]{} & \makebox[.15in]{} & 
\makebox[.15in]{} & \makebox[.15in]{} \\ 
&&&&&&&&&&&&\\
\hline
\end{tabular}

\vspace{0.2in}
%***********************************************************
% solution
% CT: [12-5]	CI: [4-2]	CO: [1-0]
%***********************************************************

\newpage
\section*{Part 2}
For the given physical address, indicate the cache entry accessed
and the cache byte value returned {\bf in hex}.  
Indicate whether a cache miss occurs.  

If there is a cache miss, enter ``-'' for ``Cache Byte returned''.


{\bf Physical address}:  {\tt 0E34}

\begin{choice}

\item Physical address format (one bit per box)\\
{\small
\begin{tabular} {ccccccccccccc}
\makebox[.15in]{12} &
\makebox[.15in]{11} & \makebox[.15in]{10} & 
\makebox[.15in]{9} & \makebox[.15in]{8} & 
\makebox[.15in]{7} & \makebox[.15in]{6} & 
\makebox[.15in]{5} & \makebox[.15in]{4} & 
\makebox[.15in]{3} & \makebox[.15in]{2} & 
\makebox[.15in]{1} & \makebox[.15in]{0} \\ 
\end{tabular} 
}

\begin{tabular} {|c|c|c|c|c|c|c|c|c|c|c|c|c|}
\hline
\makebox[.15in]{} & 
\makebox[.15in]{} & \makebox[.15in]{} & \makebox[.15in]{} & \makebox[.15in]{} & 
\makebox[.15in{}] & \makebox[.15in]{} & \makebox[.15in]{} & \makebox[.15in]{} & 
\makebox[.15in]{} & \makebox[.15in]{} & \makebox[.15in]{} & \makebox[.15in]{}\\ 
\hline
\end{tabular}

\vspace*{.5\baselineskip}
\item Physical memory reference \\
\vspace{.1in} \\
\begin{tabular}{|l|l|}
\hline
Parameter & {    Value    } \\
\hline
\hline
Byte offset & 0x\\
\hline
Cache Index & 0x\\
\hline
Cache Tag & 0x\\
\hline
Cache Hit? (Y/N) & \\
\hline
Cache Byte returned & 0x\\
\hline
\end{tabular}

\end{choice}

%***********************************************************
% solution
%
% A. 0 1110 0011 0100
%
% B.
%    	CO:		0x0
%	CI:		0x5
%	CT:		0x71
%	cache hit?	Y
%	cache byte?	0x0B
%***********************************************************
