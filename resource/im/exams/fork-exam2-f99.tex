%%COMMENT:8:5:Process control:
\begin{problem}{5}

Consider the following C program:

\begin{verbatim}
  #include <sys/wait.h>

  main() {
    int status;
  
    printf("%s\n", "Hello");
    printf("%d\n", !fork());

    if(wait(&status) != -1)
      printf("%d\n", WEXITSTATUS(status));

    printf("%s\n", "Bye");

    exit(2);
  }
\end{verbatim}

Recall the following:
\begin{itemize}
\item Function {\tt fork} returns 0 to the child process and the child's process Id to the parent.
\item Function {\tt wait} returns {\tt -1} when there is an error, e.g.,
when the executing process has no child.
\item Macro {\tt WEXITSTATUS} extracts the exit status of the
terminating process.
\end{itemize}


What is a valid output of this program? 
{\em Hint: there are several correct solutions.}

%Answer
% Hello	
% 1		0		1  
% Bye   or 	1	or	0 
% 0    		Bye		Bye  
% 2  
% Bye
\end{problem}
