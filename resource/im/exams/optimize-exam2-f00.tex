%%COMMENT:5:8:Program optimization:
\section*{Performance Optimization}

The following problem concerns optimizing a procedure for maximum
performance on an Intel Pentium III.  Recall the following performance
characteristics of the functional units for this machine:
\begin{center}
\begin{tabular}{|l|r|r|}
\hline
Operation & Latency & Issue Time \\
\hline
Integer Add & 1 & 1 \\
Integer Multiply & 4 & 1 \\
Integer Divide & 36 & 36 \\
Floating Point Add & 3 & 1 \\
Floating Point Multiply & 5 & 2 \\
Floating Point Divide & 38 & 38 \\
Load or Store (Cache Hit) & 1 & 1 \\
\hline
\end{tabular}
\end{center}

You've just joined a programming team that is trying to develop the
world's fastest factorial routine.
Starting with recursive factorial,
they've converted the code to use iteration:
\begin{verbatim}
int fact(int n)
{
  int i;
  int result = 1;

  for (i = n; i > 0; i--)
    result = result * i;

  return result;
}
\end{verbatim}
By doing so, they have reduced the number of cycles per element (CPE)
for the function from around $63$ to around $4$ (really!).
Still, they would
like to do better.

\newpage

\begin{problem}{8}

One of the programmers heard about loop unrolling.  He generated the
following code:
\begin{verbatim}
int fact_u2(int n)
{
  int i;
  int result = 1;

  for (i = n; i > 0; i-=2) {
    result = (result * i) * (i-1);
  }

  return result;
}
\end{verbatim}
Unfortunately, the team has discovered that this code
returns 0 for some values of argument {\tt n}.
\begin{choice}
\item
For what values of {\tt n} will \verb@fact_u2@ and \verb@fact@ return
different values?  \shortanswer{.5}
%% It will return 0 whenever n is odd.
\item
Show how to fix \verb@fact_u2@ so that its behavior is identical to
{\tt fact}.  [Hint: there is a special trick for this procedure that
involves modifying just a single character.]
%% Change loop test to i > 1
\item
Benchmarking \verb@fact_u2@ shows no improvement in performance.  How
would you explain that?
%% Performance is limited by the 4 cycle latency of integer multiplication.
\shortanswer{.75}
\item
You modify the line inside the loop to read:
\begin{verbatim}
    result = result * (i * (i-1));
\end{verbatim}
To everyone's astonishment, the
measured performance now has a CPE of $2.5$.  How do you
explain this performance improvement?
\shortanswer{.75}
%% The multiplication i * (i-1) can overlap with the multiplication
%% by result from the previous iteration.
\end{choice}

\end{problem}
