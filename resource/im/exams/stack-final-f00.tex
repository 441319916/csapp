%%COMMENT:3:6:Stack discipline:
This next problem will test your understanding of stack frames.  It is
based on the following recursive C function:

{\small\begin{verbatim}
int silly(int n, int *p)
{
    int val, val2;

    if (n > 0)
        val2 = silly(n << 1, &val);
    else
        val = val2 = 0;

    *p = val + val2 + n;

    return val + val2;
}
\end{verbatim}}

This yields the following machine code:

{\small\begin{verbatim}
silly:
        pushl %ebp
        movl %esp,%ebp
        subl $20,%esp
        pushl %ebx
        movl 8(%ebp),%ebx
        testl %ebx,%ebx
        jle .L3
        addl $-8,%esp
        leal -4(%ebp),%eax
        pushl %eax
        leal (%ebx,%ebx),%eax
        pushl %eax
        call silly
        jmp .L4
        .p2align 4,,7
.L3:
        xorl %eax,%eax
        movl %eax,-4(%ebp)
.L4:
        movl -4(%ebp),%edx
        addl %eax,%edx
        movl 12(%ebp),%eax
        addl %edx,%ebx
        movl %ebx,(%eax)
        movl -24(%ebp),%ebx
        movl %edx,%eax
        movl %ebp,%esp
        popl %ebp
        ret
\end{verbatim}}

\newpage

\begin{problem}{6}

\begin{choice}

\item
Is the variable {\tt val} stored on the stack? If so, at what byte offset
(relative to {\tt \%ebp}) is it stored, and why is it necessary to store it
on the stack?
%answer: yes, -4, Need to pass pointer to it to recursive call

\vspace{1.0 in}

\item
Is the variable {\tt val2} stored on the stack? If so, at what byte offset
(relative to {\tt \%ebp}) is it stored, and why is it necessary to store it
on the stack?
%answer: no

\vspace{1.0 in}

\item 
What (if anything) is stored at {\tt -24(\%ebp)}?
If something is stored there, why is it necessary to store it?

\vspace{0.5 in}
%answer: the value of %ebx is saved here, because %ebx is a
%callee-save register.


\item 
What (if anything) is stored at {\tt -8(\%ebp)}?
If something is stored there, why is it necessary to store it?
%answer: nothing is stored here.

\vspace{0.5 in}

\end{choice}

\end{problem}
