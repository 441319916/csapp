%%COMMENT:6:5:Low-level cache access:
\begin{problem}{5}
The following problem concerns basic cache lookups.
\end{problem}

\begin{itemize}
\item The memory is byte addressable.
\item Memory accesses are to {\bf 1-byte words} (not 4-byte words).
\item Physical addresses are 12 bits wide.
\item The cache is 4-way set associative, with a 2-byte block size and 32 total lines.
\end{itemize}

In the following tables, {\bf all numbers are given in hexadecimal}.
The contents of the cache are as follows:

\begin{center}
\small
\begin{tabular}{|c||c c|c c||c c|c c||c c|c c||c c|c c|}
\hline
\multicolumn{17}{|c|}{4-way Set Associative Cache}\\
\hline
\makebox[.2in]{Index} & \makebox[.2in]{Tag} & \makebox[.2in]{Valid} &
\makebox[.3in]{Byte 0} & \makebox[.3in]{Byte 1} &
\makebox[.2in]{Tag} & \makebox[.2in]{Valid} &
\makebox[.3in]{Byte 0} & \makebox[.3in]{Byte 1} &
\makebox[.2in]{Tag} & \makebox[.2in]{Valid} &
\makebox[.3in]{Byte 0} & \makebox[.3in]{Byte 1} &
\makebox[.2in]{Tag} & \makebox[.2in]{Valid} &
\makebox[.3in]{Byte 0} & \makebox[.3in]{Byte 1} \\ 
\hline
\hline
0  &  29 & 0 & 34 & 29  &  87 & 0 & 39 & AE  &  7D & 1 & 68 & F2  &  8B & 1 & 64 & 38 \\
1  &  F3 & 1 & 0D & 8F  &  3D & 1 & 0C & 3A  &  4A & 1 & A4 & DB  &  D9 & 1 & A5 & 3C \\
2  &  A7 & 1 & E2 & 04  &  AB & 1 & D2 & 04  &  E3 & 0 & 3C & A4  &  01 & 0 & EE & 05 \\
3  &  3B & 0 & AC & 1F  &  E0 & 0 & B5 & 70  &  3B & 1 & 66 & 95  &  37 & 1 & 49 & F3 \\
4  &  80 & 1 & 60 & 35  &  2B & 0 & 19 & 57  &  49 & 1 & 8D & 0E  &  00 & 0 & 70 & AB \\
5  &  EA & 1 & B4 & 17  &  CC & 1 & 67 & DB  &  8A & 0 & DE & AA  &  18 & 1 & 2C & D3 \\
6  &  1C & 0 & 3F & A4  &  01 & 0 & 3A & C1  &  F0 & 0 & 20 & 13  &  7F & 1 & DF & 05 \\
7  &  0F & 0 & 00 & FF  &  AF & 1 & B1 & 5F  &  99 & 0 & AC & 96  &  3A & 1 & 22 & 79 \\
\hline
\end{tabular}
%\end{minipage}
\end{center}

\section*{Part 1}

The box below shows the format of a physical address.  Indicate
(by labeling the diagram) the fields that would be used to determine
the following:\\

\begin{tabular}{cl}
{\em CO} & The block offset within the cache line\\
{\em CI} & The cache index\\
{\em CT} & The cache tag\\
\end{tabular}

\vspace{0.2in}
{\small
\begin{tabular} {cccccccccccc} 
\makebox[.15in]{11} & \makebox[.15in]{10} &
\makebox[.15in]{9} & \makebox[.15in]{8} & 
\makebox[.15in]{7} & \makebox[.15in]{6} & 
\makebox[.15in]{5} & \makebox[.15in]{4} & 
\makebox[.15in]{3} & \makebox[.15in]{2} & 
\makebox[.15in]{1} & \makebox[.15in]{0} \\ 
\end{tabular} 
}

\begin{tabular} {|c|c|c|c|c|c|c|c|c|c|c|c|}
\hline
\makebox[.15in]{} & \makebox[.15in]{} & 
\makebox[.15in]{} & \makebox[.15in]{} & \makebox[.15in]{} & \makebox[.15in]{} & 
\makebox[.15in]{} & \makebox[.15in]{} & \makebox[.15in]{} & \makebox[.15in]{} & 
\makebox[.15in]{} & \makebox[.15in]{} \\ 
&&&&&&&&&&&\\
\hline
\end{tabular}

\vspace{0.2in}
%***********************************************************
% solution
% CT: [11-4]	CI: [3-1]	CO: [0]
%***********************************************************

\newpage
\section*{Part 2}
For the given physical address, indicate the cache entry accessed
and the cache byte value returned {\bf in hex}.  
Indicate whether a cache miss occurs.  

If there is a cache miss, enter ``-'' for ``Cache Byte returned''.


{\bf Physical address}:  {\tt 3B6}

\begin{subproblem}

\item Physical address format (one bit per box)\\
\begin{tabular} {cccccccccccc}
\makebox[.15in]{11} & \makebox[.15in]{10} & 
\makebox[.15in]{9} & \makebox[.15in]{8} & 
\makebox[.15in]{7} & \makebox[.15in]{6} & 
\makebox[.15in]{5} & \makebox[.15in]{4} & 
\makebox[.15in]{3} & \makebox[.15in]{2} & 
\makebox[.15in]{1} & \makebox[.15in]{0} \\ 
\end{tabular} 

\begin{tabular} {|c|c|c|c|c|c|c|c|c|c|c|c|}
\hline
\makebox[.15in]{} & \makebox[.15in]{} & \makebox[.15in]{} & \makebox[.15in]{} & 
\makebox[.15in{}] & \makebox[.15in]{} & \makebox[.15in]{} & \makebox[.15in]{} & 
\makebox[.15in]{} & \makebox[.15in]{} & \makebox[.15in]{} & \makebox[.15in]{}\\ 
\hline
\end{tabular}

\vspace*{.5\baselineskip}

\item Physical memory reference
\vspace{.1in}

\begin{tabular}{|l|l|}
\hline
Parameter & Value\\
\hline
\hline
Cache Offset (CO) & 0x\\
\hline
Cache Index (CI) & 0x\\
\hline
Cache Tag (CT) & 0x\\
\hline
Cache Hit? (Y/N) &\\
\hline
Cache Byte returned & 0x\\
\hline
\end{tabular}

\end{subproblem}


\answer{
1. 	CT: [11-4]   CI: [1-3]   CO: [0]

2.
   A.	0011 1011 0110

   B.	CO:		0x0
	CI:		0x3
	CT:		0x3B
	cache hit?	Y
	cache byte?	0x66
}
%***********************************************************
% solution
%
% A. 	0011 1011 0110
%
% B.
%    	CO:		0x0
%	CI:		0x3
%	CT:		0x3B
%	cache hit?	Y
%	cache byte?	0x66
%***********************************************************

