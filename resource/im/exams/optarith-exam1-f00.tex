%%COMMENT:3:7:Mapping of arithment operations to assembly:
\begin{problem}{7}

\noindent
Match each of the assembler routines on the left with the equivalent
C function on the right.

\vspace{.2in}
\begin{minipage}[l]{2.5in}

\begin{verbatim}
foo1:
     pushl %ebp
     movl %esp,%ebp
     movl 8(%ebp),%eax
     sall $4,%eax
     subl 8(%ebp),%eax
     movl %ebp,%esp
     popl %ebp
     ret

foo2:
     pushl %ebp
     movl %esp,%ebp
     movl 8(%ebp),%eax
     testl %eax,%eax
     jge .L4
     addl $15,%eax
.L4:
     sarl $4,%eax
     movl %ebp,%esp
     popl %ebp
     ret

foo3:
     pushl %ebp
     movl %esp,%ebp
     movl 8(%ebp),%eax
     shrl $31,%eax
     movl %ebp,%esp
     popl %ebp
     ret


\end{verbatim}




\end{minipage}\makebox[1in]{}\begin{minipage}[r]{2.5in}
\begin{verbatim}
int choice1(int x)
{
    return (x < 0);
}


int choice2(int x)
{
    return (x << 31) & 1;
}


int choice3(int x)
{
    return 15 * x;
}


int choice4(int x)
{
    return (x + 15) /4
}


int choice5(int x)
{
    return x / 16;
}


int choice6(int x)
{
    return (x >> 31);    
}

\end{verbatim}
\bf{Fill in your answers here:}\\
\normalfont
\noindent foo1 corresponds to choice \rule{45pt}{.1pt}.\\
% Answer: Choice 3: return 15*x;
foo2 corresponds to choice \rule{45pt}{.1pt}.\\
% Answer: Choice 5: return x / 16;
foo3 corresponds to choice \rule{45pt}{.1pt}.\\
% Answer: Choice 6 return (x >> 31);    
\end{minipage}


\end{problem}
