%%COMMENT:3:8:Stack discipline (buffer overflow):
The next problem concerns the following C code:

\begin{alltt}
{\em/* copy string x to buf */}
void foo(char *x) \{
  int buf[1];
  strcpy((char *)buf, x);
\}

void callfoo() \{
  foo("abcdefghi");
\}
\end{alltt} 

Here is the corresponding machine code on a Linux/x86 machine:
\begin{alltt}
080484f4 <foo>:
080484f4:	55             	pushl  %ebp
080484f5:	89 e5          	movl   %esp,%ebp
080484f7:	83 ec 18       	subl   $0x18,%esp
080484fa:	8b 45 08       	movl   0x8(%ebp),%eax
080484fd:	83 c4 f8       	addl   $0xfffffff8,%esp
08048500:	50             	pushl  %eax
08048501:	8d 45 fc       	leal   0xfffffffc(%ebp),%eax
08048504:	50             	pushl  %eax
08048505:	e8 ba fe ff ff 	call   80483c4 <strcpy>
0804850a:	89 ec          	movl   %ebp,%esp
0804850c:	5d             	popl   %ebp
0804850d:	c3             	ret    

08048510 <callfoo>:
08048510:	55             	pushl  %ebp
08048511:	89 e5          	movl   %esp,%ebp
08048513:	83 ec 08       	subl   $0x8,%esp
08048516:	83 c4 f4       	addl   $0xfffffff4,%esp
08048519:	68 9c 85 04 08 	pushl  $0x804859c   {\em# push string address}
0804851e:	e8 d1 ff ff ff 	call   80484f4 <foo>
08048523:	89 ec          	movl   %ebp,%esp
08048525:	5d             	popl   %ebp
08048526:	c3             	ret    
\end{alltt}

\newpage
\begin{problem}{8} 
This problem tests your understanding of the stack discipline
and byte ordering. Here are some notes to help you work the problem:
\begin{itemize}
\item {\tt strcpy(char *dst, char *src)} copies the
string at address {\tt src} (including the terminating 
'{\tt \symbol{92}0}' character) to address {\tt dst}. It does {\bf not}
check the size of the destination buffer.

\item Recall that Linux/x86 machines are Little Endian.

\item You will need to know the hex values of the following characters:

\begin{tabular}{|l|l||l|l|}
\hline
Character & Hex value & Character & Hex value \\
\hline
\hline
'a' & 0x61 & 'f' & 0x66 \\ 
'b' & 0x62 & 'g' & 0x67 \\ 
'c' & 0x63 & 'h' & 0x68 \\ 
'd' & 0x64 & 'i' & 0x69 \\ 
'e' & 0x65 & '{\tt \symbol{92}0}' & 0x00 \\ 
\hline
\end{tabular}
\end{itemize}

Now consider what happens on a Linux/x86 machine
when {\tt callfoo} calls {\tt foo} with the
input string ``{\tt abcdefghi}''. 
\begin{choice}
\item 
List the contents of the following
memory locations immediately after {\tt strcpy} returns to {\tt foo}.
Each answer should be an unsigned 4-byte integer expressed as 8 hex
digits.
\begin{verbatim}

buf[0] = 0x____________________

buf[1] = 0x____________________

buf[2] = 0x____________________

\end{verbatim}

%Solution: 
%	buf[0] = 0x64636261
%       buf[1] = 0x68676665
%	buf[2] = 0x08040069	

\item Immediately {\bf before} the {\tt ret} instruction at address 
{\tt 0x0804850d} executes, what is the value of the frame pointer 
register \verb:%ebp:? 
\begin{verbatim}

%ebp  = 0x____________________

\end{verbatim}
% Solution: ebp = 0x68676665

\item Immediately {\bf after} the {\tt ret} instruction at address
{\tt 0x0804850d} executes, what is the value of the program counter
register \verb:%eip:?
\begin{verbatim}

%eip  = 0x____________________

\end{verbatim}
% Solution: eip = 0x08040069

\end{choice}
\end{problem}




