%%COMMENT:3:14:Array layout and access:
\begin{problem}{14}

Consider the source code below, used to keep track of the rooms
currently reserved in a family-run hotel. Each entry in the {\tt
residents} array stores a name of the customer reserving the
room. {\tt FLOORS} represents the number of floors in the hotel.  {\tt
ROOMS} represents the number of rooms per floor.  Both are constants
declared with \verb@#define@. LEN, the maximum number of bytes
allocated for a name, is defined to be 12.

\begin{verbatim}
char residents[FLOORS][ROOMS][LEN];

void
reserve_room(int floor, int room, char *custname)
{
        strcpy(residents[floor][room], custname);
}


\end{verbatim}

The assembly code for the function {\tt reserve\_room} looks like this:

\begin{verbatim}
reserve_room:
        pushl %ebp
        movl %esp,%ebp
        movl 12(%ebp),%eax
        movl 16(%ebp),%edx
        pushl %edx
        movl 8(%ebp),%edx
        sall $4,%edx
        subl 8(%ebp),%edx
        leal (%eax,%eax,2),%eax
        leal residents(,%eax,4),%eax
        leal (%eax,%edx,4),%edx
        pushl %edx
        call strcpy
        movl %ebp,%esp
        popl %ebp
        ret
\end{verbatim}

\begin{choice}
\item What is the value of {\tt ROOMS}?

\vspace{.4 in}
%% Answer: 5

\item Due to a strange bug, the program accesses {\tt
residents[0][1][-2]}. What value is actually being accessed? (Express
your answer as an \emph{integer} \emph{triplet (-,-,-)}. You may 
assume that FLOORS and ROOMS are both greater than 1)

\vspace{.4 in}
%% Answer: (0, 0, 10)

\end{choice}
\end{problem}

\newpage

C. The programmer realizes that this implementation is wasteful of
memory. Successive fires in several memory chip factories in Taiwan
drive up memory prices and finally convince him to improve the memory
efficiency of his implementation to maintain the competitiveness
of the family hotel. 

The declaration of {\tt residents} is changed to be a two dimensional
array of pointers to character strings (names).  The new code
allocates memory for customer names only for those rooms that are
actually reserved. Otherwise, {\tt residents[f][r]} stores a {NULL}
pointer.  {\bf For simplicity, assume there is no storage overhead
due to {\tt malloc}}.

The new declaration looks like this:

\begin{verbatim}
char *residents[FLOORS][ROOMS];

void
reserve_room(int floor, int room, char *custname)
{
    residents[floor][room] = malloc(LEN);
    strcpy(residents[floor][room], custname);	
}
\end{verbatim}

After a few months. The programmer goes back to review the memory
savings of his improved scheme. During that period, the hotel was 20\%
reserved. The programmer is delighted because the savings are found to
be 168 bytes!  How many floors does this hotel have? (that is, what is
the value of FLOORS?)

\vspace{.4 in}
%
%
%% Answer: 6
%
% One way to think of this problem:
% There are five rooms per floor. The first scheme allocates 12*5 bytes
% per floor.
% The second scheme allocates 4 bytes for empty rooms (null ptr) and
% *4+12=16* bytes for reserved rooms.
% With 20% utilization, that means 4*4 + 1*16 = 32 bytes per floor.
%
% Thus, the second scheme saves 28 bytes per floor, or 168 bytes per
% 6 floors.
%


